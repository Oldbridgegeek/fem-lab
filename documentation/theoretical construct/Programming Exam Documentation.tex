\documentclass[a4paper,12pt]{article}
\usepackage{amsmath}
\usepackage{bm}
\begin{document}

\title{Programming Exam Documentation}

\author{Marcel Dürr \and Enes Witwit}

\section{Introduction}



\section{Weak Formulation}

In order to apply some of the results from the lecture, we need to derive the weak formulation of the given problem 
\begin{align}
\mbox{Find } u\in C^2(\Omega)\mbox{ :}-\Delta u+u &= \cos(\pi x)\cos(\pi y) &&\mbox{in } \Omega \\
\partial _n u &= 0 &&\mbox{on } \partial\Omega.
\end{align}
Multiplying with an arbitary $v\in C^2(\Omega)$ and integrating over $\Omega$ gives us
\[-\int _\Omega \Delta uv \,\mbox{d} \bm{x} + \int _\Omega uv \,\mbox{d} \bm{x} = \int _ \Omega fv\,\mbox{d} \bm{x}\]
where $f=\cos(\pi x)\cos(\pi y)$ and $\bm{x}=(x,y)$. Using Green's first formula and (2) we can obtain the weak formulation
\[\int _\Omega \nabla u\nabla v \,\mbox{d} \bm{x} + \int _\Omega uv \,\mbox{d} \bm{x} = \int _ \Omega fv\,\mbox{d} \bm{x}\]
From now on, we will denote the left hand side of the equation by $a(u,v)$ and the right hand side by $F(v)$. Thus, we obtain the weak formulation
\begin{equation}
\mbox{Find }u\in H^1(\Omega) \mbox{ :} \quad a(u,v)=F(v)\quad \forall v \in H^1(\Omega)
\end{equation}

\section{Existence and Uniqueness of a Solution}

We can already see, that our bilinear form $a(\cdotp ,\cdot)$ is the inner product associated with the norm on our function space $H^1(\Omega)$. We want to use the Riesz representation theorem to prove existence and uniqueness of a solution. In order to do so, it remains to show that our functional $F(\cdot)$ is linear and bounded. Linearity follows from the properties of integration. Using Hoelder's inequality, we show that
\begin{align*}
F(u)&=\|fu\|_{L^1(\Omega)}\\
	&\stackrel{\tiny\mbox{Hld.}}{\le} \|f\|_{L^2(\Omega)}\|u\|_{L^2(\Omega)}\\
	&\le \|1\|_{L^2(\Omega)}\ \big(\|u\|_{L^2(\Omega)}+\|u\|_{L^2(\Omega)}\big)\\
	&\le c\ \|u\|_{H^1(\Omega)}\mbox{,}
\end{align*}
where c depends on our domain $\Omega$. For our case, we have $\Omega=[0,1]^2$, in particular this means that $\Omega$ is bounded and our constant $c$ is finite. Therefore, $F(\cdot)$ is a bounded, linear functional and we can apply the Riesz representation theorem.

\section{Finding the Analytical Solution}

Now that we know that a unique solution exists, we want to actually compute it. We will use the ansatz $u=C\cos(\pi x)\cos(\pi y)$, with its gradient $\Delta u=2\pi^2 u$. Inserting in (1) gives us
\begin{equation}
-2C\pi^2 \cos(\pi x)\cos(\pi y) + C\cos(\pi x)\cos(\pi y)=\cos(\pi x)\cos(\pi y)
\end{equation}
\begin{align}
&\Rightarrow - 2C\pi^2 + C =1\\
&\Leftrightarrow C = \frac{1}{1-2\pi^2}
\end{align}
This leaves us with the solution $u=\frac{1}{1-2\pi^2}\cos(\pi x)\cos(\pi y)$.

\end{document}